\documentclass{article}

\usepackage[utf8x]{inputenc}
\usepackage[frenchb]{babel}
\usepackage[T1]{fontenc}
\usepackage{lmodern}
\usepackage{fullpage}
\usepackage{graphicx}
\usepackage{epstopdf}
\usepackage{caption}
\usepackage{subcaption}
\usepackage{multirow}

% Math symbols
\usepackage{amsmath}
\usepackage{amssymb}
\usepackage{amsthm}

% Numbers and units
\usepackage[squaren, Gray]{SIunits}
\usepackage{sistyle}
\usepackage[autolanguage]{numprint}
%\usepackage{numprint}
\newcommand\si[2]{\numprint[#2]{#1}}
\newcommand\np[1]{\numprint{#1}}

\DeclareMathOperator{\pgcd}{pgcd} % use \dif instead

\DeclareMathOperator{\newdiff}{d} % use \dif instead
\newcommand{\dif}{\newdiff\!}
\newcommand{\fpart}[2]{\frac{\partial #1}{\partial #2}}
\newcommand{\ffpart}[2]{\frac{\partial^2 #1}{\partial #2^2}}
\newcommand{\fdpart}[3]{\frac{\partial^2 #1}{\partial #2\partial #3}}
\newcommand{\fdif}[2]{\frac{\dif #1}{\dif #2}}
\newcommand{\ffdif}[2]{\frac{\dif^2 #1}{\dif #2^2}}
\newcommand{\constant}{\ensuremath{\mathrm{cst}}}

% cfr http://en.wikibooks.org/wiki/LaTeX/Colors
\usepackage{color}
\usepackage[usenames,dvipsnames,svgnames,table]{xcolor}
\definecolor{dkgreen}{rgb}{0.25,0.7,0.35}
\definecolor{dkred}{rgb}{0.7,0,0}

\usepackage{listings}
\lstset{
  numbers=left,
  numberstyle=\tiny\color{gray},
  basicstyle=\rm\small\ttfamily,
  keywordstyle=\bfseries\color{dkred},
  frame=single,
  commentstyle=\color{gray}=small,
  stringstyle=\color{dkgreen},
  %backgroundcolor=\color{gray!10},
  %tabsize=2,
  rulecolor=\color{black!30},
  %title=\lstname,
  breaklines=true,
  framextopmargin=2pt,
  framexbottommargin=2pt,
  extendedchars=true,
  inputencoding=utf8x
}
\lstset{language={matlab}}

\title{Devoir 1}
\author{Arnaud Cerckel \and Benoît Legat \and
Nicolas Stevens \and Harold Taeter}

\begin{document}

\maketitle

\section*{Question 1}
Soient les matrices tridiagonales symétriques
$$T_n =
\begin{pmatrix}
\alpha_1 & \beta_2 & & \\
\beta_2 & \ddots & \ddots & \\
 & \ddots & \ddots & \beta_n \\
 & & \beta_n & \alpha_n
\end{pmatrix}
$$
avec $\beta_2 \hdots \beta_n$ non nuls.

On cherche à montrer que les polynômes orthogonaux définis par $p_n(x)$ = det($x I_n - T_n$) satisfont la récurrence $$p_n(x) = (x-\alpha_n)p_{n-1}(x) - \beta_n^2 p_{n-2}(x).$$
On va simplement calculer le déterminant de la matrice $x I_n - T_n$ par la méthode des cofacteurs. On a
$$x I_n - T_n =
\begin{pmatrix}
x-\alpha_1 & -\beta_2 & & \\
-\beta_2 & \ddots & \ddots & \\
 & \ddots & \ddots & -\beta_n \\
 & & -\beta_n & x-\alpha_n
\end{pmatrix}.
$$
Utilisons la méthodes des cofacteurs suivant les éléments de la dernière ligne de la matrice. On a alors $$\text{det}(x I_n - T_n) = (-1)^{n+n}(x-\alpha_n)\text{det}(x I_{n-1} - T_{n-1}) + (-1)^{n+n-1}(-\beta_n)\text{det}(S_n)$$ où $S_n$ est défini par
$$S_n =
\begin{pmatrix}
x-\alpha_1 & -\beta_2 & & \\
-\beta_2 & \ddots & \ddots & \\
 & \ddots & x-\alpha_{n-2} & 0 \\
 & & -\beta_{n-1} & -\beta_n
\end{pmatrix}.
$$
Si on calcule le déterminant de la matrice $S_n$ par la méthode des cofacteurs suivant la dernière colonne, on obtient $$\text{det}(S_n) = (-1)^{n-1+n-1}(-\beta_n)\text{det}(x I_{n-2} - T_{n-2}).$$
Finalement, on obtient donc l'égalité suivante : $$\text{det}(x I_n - T_n) = (-1)^{2n}(x-\alpha_n)\text{det}(x I_{n-1} - T_{n-1}) + (-1)^{4n-3}(-\beta_n)^2\text{det}(x I_{n-2} - T_{n-2}).$$
C'est-à-dire $$p_n(x) = (x-\alpha_n)p_{n-1}(x) - \beta_n^2 p_{n-2}(x),$$ la relation qu'il fallait vérifier.

\section*{Question 2}
Commençons par montrer que $p_n$ est un polynôme de degré $n$.
On peut le montrer par induction à l'aide de l'équation de récurrence
\[ p_n(x) = (x - \alpha_n)p_{n-1}(x) - \beta_n^2p_{n-2}(x). \]
\begin{itemize}
  \item C'est évident pour $p_0(x)$ qui vaut 1 pour que la récurrence
    marche.
  \item Si $p_{n-1}(x)$ est de degré $n-1$ et $p_{n-2}(x)$ est de degré
    $n-2$, $p_n(x)$ qui vaut
    \[ p_n(x) = xp_{n-1}(x) - (\alpha_np_{n-1}(x) + \beta_n^2p_{n-2}(x)) \]
    est la somme entre un polynôme de degré $n$ et un polynôme de
    degré au plus $n-1$ ce qui fait un polynôme de degré $n$.
\end{itemize}

Remarquons à présent à l'aide de l'équation de récurrence
\[ p_n(x) = (x - \alpha_n)p_{n-1}(x) - \beta_n^2p_{n-2}(x) \]
que si $(x-x_0)$ divise $p_n$ et $p_{n-1}$, alors il divise
$p_{n-2}$ et que s'il divise $p_{n-1}$ et $p_{n-2}$, alors il divise $p_n$.
On voit alors que $\pgcd(p_n, p_{n-1}) = \pgcd(p_{n-1}, p_{n-2})$.

Comme $p_0(x) = 1$, $\pgcd(p_1,p_0) = p_0$ et donc, par transitivité,
$\pgcd(p_n, p_{n-1}) = p_0 = 1$.
On sait donc que $p_n$ et $p_{n-1}$ n'ont pas de racine commune.

On remarque que la matrice obtenue en supprimant
la dernière ligne et la dernière colonne de $T_n$ est $T_{n-1}$.
Par le théorème 1, on sait que leur valeurs propres sont entrelacées.
Comme leurs valeurs propres sont les racines des polynômes $p_n$ et $p_{n-1}$.
On peut dire que leurs racines sont entrelacées.

Du coup, si $p_i$ a une racine multiple $r$, $p_{i+1}$ et $p_{i-1}$,
s'ils existent ont nécessairement cette racine aussi car on aura alors
un $k$ tel que $\lambda_k = \lambda_{k+1} = r$ et donc $\mu_{k} = r$.
Pourtant, on sait que $p_i$ et $p_{i+1}$ n'ont pas de racine commune,
pareil pour $p_i$ et $p_{i-1}$.
On en conclut que $p_i$ n'a pas de racine multiple.

\begin{enumerate}
  \item Commençons par montrer que $f_0(\alpha)f_0(\beta) \neq 0$.
    Comme $\alpha = -\infty$ et $\beta = \infty$ et que $f_0$ est un polynôme,
    il faut que $f_0$ soit le polynôme nul.
    Pourtant, on montré que $p_n$ était de degré $n$ et $p_0 = 1$ donc
    c'est impossible.
  \item Montrons à présent que si $f_i(\xi) = 0$ avec $1 \leq i \leq n-1$,
    alors $f_{i-1}(\xi)f_{i+1}(\xi) < 0$.
    L'équation de récurrence nous donne
    $f_{i+1}(\xi) = -\beta_n^2f_{i-1}(\xi)$ du coup
    \[ f_{i-1}(\xi)f_{i+1}(\xi) = -(\beta_nf_{i-1}(\xi))^2 < 0. \]
    car $\beta_n \neq 0$ et comme on l'a montré, $f_i$ et $f_{i-1}$ ne peuvent
    pas avoir de racine commune donc $f_{i-1}(\xi) \neq 0$.
  \item Montrons maintenant que si $f_0(\xi) = 0$,
    alors $f_0'(\xi)f_1(\xi) > 0$.
    On a
    \begin{align*}
      f_0'(x) & = p_n'(x)\\
              & = p_{n-1}(x) + (x-\alpha_n)p_{n-1}'(x) - \beta_n^2p_{n-2}'(x)\\
              & = f_1(x) + (x-\alpha_n) f_1'(x) - \beta_n^2 f_2'(x).
    \end{align*}
    Du coup,
    \begin{align*}
      f_0'(\xi)f_1(\xi) & = f_1^2(x) + f_1(x)
      ((\xi-\alpha_n) f_1'(x) - \beta_n^2 f_2'(x))\\
      & > 0.
    \end{align*}
  \item Le fait que $f_n(x)$ soit de signe constant est trivial car
    $f_n(x) = p_0(x) = 1$.
\end{enumerate}

\section*{Question 3}
Il suffit de réécrire le polynôme
\[ a_nx^n + a_{n-1}x^{n-1} + a_{n-2}x^{n-2} + \ldots + a_1x + a_0 \]
en
\[ ((\ldots((a_nx + a_{n-1})x + a_{n-2})x + \ldots)x + a_1)x + a_0. \]

Ce qui donne
\lstinputlisting{matlab/badass_polyval.m}

\end{document}
