\documentclass{article}

\usepackage[utf8x]{inputenc}
\usepackage[frenchb]{babel}
\usepackage[T1]{fontenc}
\usepackage{lmodern}
\usepackage{fullpage}
\usepackage{graphicx}
\usepackage{epstopdf}
\usepackage{caption}
\usepackage{subcaption}
\usepackage{multirow}

% Math symbols
\usepackage{amsmath}
\usepackage{amssymb}
\usepackage{amsthm}

% Numbers and units
\usepackage[squaren, Gray]{SIunits}
\usepackage{sistyle}
\usepackage[autolanguage]{numprint}
%\usepackage{numprint}
\newcommand\si[2]{\numprint[#2]{#1}}
\newcommand\np[1]{\numprint{#1}}

\DeclareMathOperator{\pgcd}{pgcd} % use \dif instead

\DeclareMathOperator{\newdiff}{d} % use \dif instead
\newcommand{\dif}{\newdiff\!}
\newcommand{\fpart}[2]{\frac{\partial #1}{\partial #2}}
\newcommand{\ffpart}[2]{\frac{\partial^2 #1}{\partial #2^2}}
\newcommand{\fdpart}[3]{\frac{\partial^2 #1}{\partial #2\partial #3}}
\newcommand{\fdif}[2]{\frac{\dif #1}{\dif #2}}
\newcommand{\ffdif}[2]{\frac{\dif^2 #1}{\dif #2^2}}
\newcommand{\constant}{\ensuremath{\mathrm{cst}}}
\newcommand{\bigoh}{\ensuremath{\mathcal{O}}}

% cfr http://en.wikibooks.org/wiki/LaTeX/Colors
\usepackage{color}
\usepackage[usenames,dvipsnames,svgnames,table]{xcolor}
\definecolor{dkgreen}{rgb}{0.25,0.7,0.35}
\definecolor{dkred}{rgb}{0.7,0,0}

\usepackage{listings}
\lstset{
  numbers=left,
  numberstyle=\tiny\color{gray},
  basicstyle=\rm\small\ttfamily,
  keywordstyle=\bfseries\color{dkred},
  frame=single,
  commentstyle=\color{gray}=small,
  stringstyle=\color{dkgreen},
  %backgroundcolor=\color{gray!10},
  %tabsize=2,
  rulecolor=\color{black!30},
  %title=\lstname,
  breaklines=true,
  framextopmargin=2pt,
  framexbottommargin=2pt,
  extendedchars=true,
  inputencoding=utf8x
}
\lstset{language={matlab}}

\title{Devoir 1}
\author{Arnaud Cerckel \and Benoît Legat \and
Nicolas Stevens \and Harold Taeter}

\begin{document}

\maketitle

\section*{Question 1}
Soient les matrices tridiagonales symétriques
$$T_n =
\begin{pmatrix}
\alpha_1 & \beta_2 & & \\
\beta_2 & \ddots & \ddots & \\
 & \ddots & \ddots & \beta_n \\
 & & \beta_n & \alpha_n
\end{pmatrix}
$$
avec $\beta_2 \hdots \beta_n$ non nuls.

On cherche à montrer que les polynômes orthogonaux définis par $p_n(x)$ = det($x I_n - T_n$) satisfont la récurrence $$p_n(x) = (x-\alpha_n)p_{n-1}(x) - \beta_n^2 p_{n-2}(x).$$
On va simplement calculer le déterminant de la matrice $x I_n - T_n$ par la méthode des cofacteurs. On obtient
$$x I_n - T_n =
\begin{pmatrix}
x-\alpha_1 & -\beta_2 & & \\
-\beta_2 & \ddots & \ddots & \\
 & \ddots & \ddots & -\beta_n \\
 & & -\beta_n & x-\alpha_n
\end{pmatrix}.
$$
Utilisons la méthodes des cofacteurs suivant les éléments de la dernière ligne de la matrice. On a alors $$\text{det}(x I_n - T_n) = (-1)^{n+n}(x-\alpha_n)\text{det}(x I_{n-1} - T_{n-1}) + (-1)^{n+n-1}(-\beta_n)\text{det}(S_n)$$ où $S_n$ est défini par
$$S_n =
\begin{pmatrix}
x-\alpha_1 & -\beta_2 & & \\
-\beta_2 & \ddots & \ddots & \\
 & \ddots & x-\alpha_{n-2} & 0 \\
 & & -\beta_{n-1} & -\beta_n
\end{pmatrix}.
$$
Si on calcule le déterminant de la matrice $S_n$ par la méthode des cofacteurs suivant la dernière colonne, on obtient $$\text{det}(S_n) = (-1)^{n-1+n-1}(-\beta_n)\text{det}(x I_{n-2} - T_{n-2}).$$
Finalement, on obtient donc l'égalité suivante : $$\text{det}(x I_n - T_n) = (-1)^{2n}(x-\alpha_n)\text{det}(x I_{n-1} - T_{n-1}) + (-1)^{4n-3}(-\beta_n)^2\text{det}(x I_{n-2} - T_{n-2}).$$
C'est-à-dire $$p_n(x) = (x-\alpha_n)p_{n-1}(x) - \beta_n^2 p_{n-2}(x),$$ la relation qu'il fallait vérifier.

On remarque que pour que la récurrence marche,
il faut poser $p_0(x) = 1$.

\section*{Question 2}
Commençons par montrer que $p_n$ est un polynôme de degré $n$ et que le
coefficient de $x_n$ est 1.
On peut le montrer par induction à l'aide de l'équation de récurrence
\[ p_n(x) = (x - \alpha_n)p_{n-1}(x) - \beta_n^2p_{n-2}(x). \]
\begin{itemize}
  \item C'est évident pour $p_0(x)$ qui vaut 1.
  \item Si $p_{n-1}(x)$ est de degré $n-1$ et $p_{n-2}(x)$ est de degré
    $n-2$, $p_n(x)$ qui vaut
    \[ p_n(x) = xp_{n-1}(x) - (\alpha_np_{n-1}(x) + \beta_n^2p_{n-2}(x)) \]
    est la somme entre un polynôme de degré $n$ et un polynôme de
    degré au plus $n-1$ ce qui fait un polynôme de degré $n$.
    Le coefficient devant $x^n$ sera le même que celui devant $x^{n-1}$ qui,
    par l'hypothèse de récurrence vaut 1.
\end{itemize}

Remarquons à présent à l'aide de l'équation de récurrence
\[ p_n(x) = (x - \alpha_n)p_{n-1}(x) - \beta_n^2p_{n-2}(x) \]
que si $(x-x_0)$ divise $p_n$ et $p_{n-1}$, alors il divise
$p_{n-2}$ et que s'il divise $p_{n-1}$ et $p_{n-2}$, alors il divise $p_n$.
On voit alors que $\pgcd(p_n, p_{n-1}) = \pgcd(p_{n-1}, p_{n-2})$.

Comme $p_0(x) = 1$, $\pgcd(p_1,p_0) = p_0$ et donc, par transitivité,
$\pgcd(p_n, p_{n-1}) = p_0 = 1$.
On sait donc que $p_n$ et $p_{n-1}$ n'ont pas de racine commune.

On remarque que la matrice obtenue en supprimant
la dernière ligne et la dernière colonne de $T_n$ est $T_{n-1}$.
Par le théorème 1, on sait que leur valeurs propres sont entrelacées.
Comme leurs valeurs propres sont les racines des polynômes $p_n$ et $p_{n-1}$, on peut dire que leurs racines sont entrelacées.

% A REFORMULER MAN 
Du coup, si $p_i$ a une racine multiple $r$, $p_{i+1}$ et $p_{i-1}$,
s'ils existent ont nécessairement cette racine aussi car on aura alors
un $k$ tel que $\lambda_k = \lambda_{k+1} = r$ et donc $\mu_{k} = r$.
Pourtant, on sait que $p_i$ et $p_{i+1}$ n'ont pas de racine commune,
pareil pour $p_i$ et $p_{i-1}$.
On en conclut que $p_i$ n'a pas de racine multiple.

\begin{enumerate}
  \item Commençons par montrer que $f_0(\alpha)f_0(\beta) \neq 0$.
    Comme $\alpha = -\infty$ et $\beta = + \infty$ et que $f_0$ est un polynôme,
    il faut que $f_0$ soit le polynôme nul.
    Pourtant, on a montré que $p_n$ était de degré $n$ et $p_0 = 1$ donc
    c'est impossible.
  \item Montrons à présent que si $f_i(\xi) = 0$ avec $1 \leq i \leq n-1$,
    alors $f_{i-1}(\xi)f_{i+1}(\xi) < 0$.
    L'équation de récurrence nous donne
    $f_{i+1}(\xi) = -\beta_n^2f_{i-1}(\xi)$ du coup
    \[ f_{i-1}(\xi)f_{i+1}(\xi) = -(\beta_nf_{i-1}(\xi))^2 < 0. \]
    car $\beta_n \neq 0$ et comme on l'a montré, $f_i$ et $f_{i-1}$ ne peuvent
    pas avoir de racine commune donc $f_{i-1}(\xi) \neq 0$.
  \item Montrons maintenant que si $f_0(\xi) = 0$,
    alors $f_0'(\xi)f_1(\xi) > 0$.

    Commençons pour cela par montrer que
    soit $\epsilon > 0$ tel que
    $f_1(x) \neq 0$ $\forall x \in [\xi - \epsilon; \xi]$,
    $f_0(\xi-\epsilon) f_1(\xi-\epsilon) < 0$.
%Stivy est pas d'accord il proteste ardemment
    \begin{proof}
      Déjà, comme les racines sont entrelacées,
      $f_1(x) \neq 0$ $\forall x \in [\xi - \epsilon; \xi]$ implique que
      $f_0(x) \neq 0$ $\forall x \in [\xi - \epsilon; \xi[$
      par la continuité des polynômes, $f_1$ et $f_0$ ne changent pas
      de signe sur $[\xi - \epsilon; \xi[$.
      De plus, en $-\infty$,
      le signe de $f_0$ est $(-1)^n$ car c'est un polynôme de
      degré $n$ et que le coefficient de $x^n$ est 1.
      De la même manière, celui de $f_1$ est $(-1)^{n-1}$,
      ils sont donc de signe opposé en $-\infty$.

      À chaque fois que $f_0$ (resp. $f_1$) traverse une racine, il change
      de signe.
      En effet, on a prouvé qu'il n'avait que des racines simple.
      S'il ne changeait pas de signe, $f'_0$ (resp. $f'_1$) en changerait.
      Il aurait donc aussi une racine en $\xi$ pourtant on sait que si une
      fonction a une racine de multiplicité $p > 0$, sa
      dérivée l'a avec une multiplicité de $p-1$.
      Ici, comme $\xi$ est une racine simple de $f_0$ (resp. $f_1$), ça ne peut
      pas être une racine de $f'_0$ (resp. $f'_1$).

      Comme les racines sont entrelacées, dans l'intervalle
      $]-\infty; \xi[$, il y a le même nombre de racines de $f_0$ que de
      racines de $f_1$.
      Ils ont donc changé de signe le même nombre de fois.
      Ils sont dès lors toujours de signe différent comme en $-\infty$.
    \end{proof}

    On sait que $p_0'(\xi)p_0(\xi - \epsilon) < 0$.
    En effet, soit $q(x)$ tel que $q(\xi) \neq 0$ et
    $p_0(x) = (x - \xi) q(x)$.
    Soit $\epsilon > 0$ tel que
    $p_1(x), q(x) \neq 0$ $\forall x \in [\xi - \epsilon, \xi]$.
    On a $p_0'(x) = q(x) + (x - \xi) q'(x)$ d'où
    $p_0(\xi - \epsilon) = -\epsilon q(\xi - \epsilon)$ et
    $p_0'(\xi) = q(\xi)$.
    Comme $q$ est un polynôme, il est continu donc comme il n'y a pas
    de racine entre $\xi - \epsilon$ et $\xi$, $q(\xi - \epsilon)$ et $q(\xi)$
    sont de même signe.
    Du même raisonnement, $p_1(\xi - \epsilon)$ et $p_1(\xi)$ sont
    de même signe.
    Comme $\epsilon > 0$, on sait alors que que
    $p_0'(\xi)$ et $p_0(\xi - \epsilon)$ sont de signe opposé.
    comme on savait que $p_0(\xi - \epsilon)$ et $p_1(\xi - \epsilon)$ sont
    de signe opposé, on a bien $p_0'(\xi) p_1(\xi) > 0$.
  \item Le fait que $f_n(x)$ soit de signe constant est trivial car
    $f_n(x) = p_0(x) = 1$.
\end{enumerate}

\section*{Question 3}
Il suffit d'utiliser la récurrence
\[ p_n(x) = (x - \alpha_n)p_{n-1}(x) - \beta_n^2 p_{n-2}(x). \]

Ce qui donne
\lstinputlisting{matlab/polyval_recurrence.m}


\subsection{Question 4}
Soit les polynômes de Tchebychev définis par la récurrence: 

\begin{eqnarray}
P_0(x)=1\\
P_1(x)=x\\
P_n(x)=2xP_{n-1}(x)-P_{n-2}(x)
\end{eqnarray}

On peut exprimer ces polynômes au moyen du déterminant des matrices $xI_n - T_n$ où $T_n$ sont des matrices tridiagonales réelles et symétriques.

On remarque que pour satisfaire cette équation et les conditions initiales $P_0$ et $P_1$, $P_n(x)$ doit avoir la forme : 
$$P_n(x) = \dfrac{1}{2} det
\left[ 
\begin{array}{cccccc}
2x & \sqrt{2} & & & &  \\
\sqrt{2} & 2x & 1 & & & \\
 & 1 & 2x & 1 & & \\
 & & \ddots & \ddots & \ddots & \\
 & & & \ddots & \ddots &  1\\
 & & & &  1 & 2x
\end{array}
\right] 
$$
En effet, cette expression vérifie l'équation de récurrence : 
$$P_n(x) = 2x \dfrac{1}{2} det
\left[ 
\begin{array}{ccccc}
2x & \sqrt{2} & & &  \\
\sqrt{2} & 2x & 1 & & \\
 & 1 & 2x & 1 & \\
 & & \ddots & \ddots & \ddots  \\
 & & & \ddots & \ddots   \\
\end{array}
\right] - \dfrac{1}{2} det
\left[ 
\begin{array}{ccccccc}
2x & \sqrt{2} & & & & & 0 \\
\sqrt{2} & 2x & 1 & & & & 0\\
 & 1 & 2x & 1 & & & 0\\
 & & \ddots & \ddots & \ddots & & 0\\
 & & & \ddots & \ddots &  1 & 0\\
 & & & &  1 & 2x & 1
\end{array}
\right] = 2xP_{n-1}(x)-P_{n-2}(x)$$
et on obtient pour les polynômes $P_1(x)$ et $P_0(x)$ : 
$$P_1(x)=\dfrac{1}{2} 2x = x $$

$$P_2(x)=2xP_1(x) - P_0(x) \Longleftrightarrow P_0(x) = 2xP_1(x) - P_2(x) = 2x \dfrac{1}{2} 2x - \dfrac{1}{2} det
\left[ 
\begin{array}{cc}
2x & \sqrt{2} \\
\sqrt{2} & 2x
\end{array}
\right] = 2x^2 - 2x^2 + 1 = 1$$

On peut décomposer cette matrice pour faire apparaître explicitement la forme recherchée, c'est-à-dire $xI_n - T_n$ : 
$$P_n = \dfrac{1}{2} det \left( 2x I_n - \left[ 
\begin{array}{cccccc}
0 & -\sqrt{2} & & & &  \\
-\sqrt{2} & 0 & -1 & & & \\
 & -1 & 0 & -1 & & \\
 & & \ddots & \ddots & \ddots & \\
 & & & \ddots & \ddots &  -1\\
 & & & &  -1 & 0
\end{array}
\right] \right) = 2^{n-1} det \left( x I_n - \underbrace{\left[ 
\begin{array}{ccccc}
0 & - \dfrac{\sqrt{2}}{2} & & &  \\
- \dfrac{\sqrt{2}}{2} & 0 & -\dfrac{1}{2} & & \\
 & -\dfrac{1}{2} & 0 & \ddots &  \\
 & & \ddots & \ddots & -\dfrac{1}{2} \\
 & & & -\dfrac{1}{2} & 0 \\
\end{array}
\right]}_{=T_n}  \right)
$$

La matrice $T_{20}$ se déduit à partir de l'expression précédente. Les valeurs propres de $T_{20}$ s'obtiennent en déterminant le noyaux de l'application linéaire $\lambda I_{20} - T_{20}$ ou autrement dit les racines du polynôme caractéristique donné par $det(\lambda I_{20} - T_{20})$. A la question précédente, nous avons implémenté une méthode permettant d'évaluer en $O(n)$ opérations la suite des polynômes $p_i$ en un point. Notons que nos polynômes $P_i$ peuvent s'écrire sous la forme : 
\begin{equation} \label{equ_tche}
P_n(x) = 2^{n-1} ((x-\alpha_n)P_{n-1}(x) - \beta_n^2 P_{n-2}(x)) = 2^{n-1} (x P_{n-1}(x) - \beta_n^2 P_{n-2}(x))
\end{equation}
Le facteur positif $2^{n-1}$ n'a aucune importance ici. En effet, pour déterminer le nombres de racines du polynôme caractéristique dans $\left[ \dfrac{1}{2} , \dfrac{3}{4} \right]$, il nous suffit d'évaluer la suite des polynômes de Tchebychev $ \{ P_0(x), P_1(x), ... , P_{20}(x) \}$ en $x=\dfrac{1}{2}$ et $x=\dfrac{3}{4}$ et d'appliquer le théorème de Sturn. Le nombre de racines sera donné par l'expression :
$$N(\dfrac{1}{2} , \dfrac{3}{4}) = V(\dfrac{1}{2}) - V(\dfrac{3}{4}) $$
Pour ce faire, on peut appliquer l'algorithme développé à la question 3 en utilisant l'équation \ref{equ_tche}. 



\begin{verbatim}
alpha =

  Columns 1 through 18

     0     0     0     0     0     0     0     0     0     0     0     0     0     0     0     0     0     0

  Columns 19 through 20

     0     0
beta =

  Columns 1 through 10

   -0.7071   -0.5000   -0.5000   -0.5000   -0.5000   -0.5000   -0.5000   -0.5000   -0.5000   -0.5000

  Columns 11 through 20

   -0.5000   -0.5000   -0.5000   -0.5000   -0.5000   -0.5000   -0.5000   -0.5000   -0.5000   -0.5000
EDU>> p = polyval_recurrence (alpha, beta, 0.5)

p =

  Columns 1 through 10

    1.0000    0.5000   -0.2500   -0.2500   -0.0625    0.0313    0.0313    0.0078   -0.0039   -0.0039

  Columns 11 through 20

   -0.0010    0.0005    0.0005    0.0001   -0.0001   -0.0001   -0.0000    0.0000    0.0000    0.0000

  Column 21

   -0.0000

 p2 =  polyval_recurrence (alpha, beta, 3/4)

p2 =

  Columns 1 through 10

    1.0000    0.7500    0.0625   -0.1406   -0.1211   -0.0557   -0.0115    0.0053    0.0069    0.0038

  Columns 11 through 20

    0.0011   -0.0001   -0.0004   -0.0002   -0.0001   -0.0000    0.0000    0.0000    0.0000    0.0000

  Column 21

   -0.0000
\end{verbatim}

On obtient ainsi $V(1/2) = 7$ et $V(3/4) = 5$. On conclut grâce au théorème de Sturn que la matrice $T_{20}$ possède 2 valeurs propre dans l'intervalle $\left[ \dfrac{1}{2} , \dfrac{3}{4} \right]$. 



\end{document}
