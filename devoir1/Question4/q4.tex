\section*{Question 4}
Soit les polynômes de Tchebychev définis par la récurrence:

\begin{align*}
  P_0(x) & = 1\\
  P_1(x) & = x\\
  P_n(x) & = 2xP_{n-1}(x) - P_{n-2}(x).
\end{align*}

On peut exprimer ces polynômes au moyen du déterminant des matrices $xI_n - T_n$ où $T_n$ sont des matrices tridiagonales réelles et symétriques.

On remarque que la récurrence ci-dessus ressemble à celle de la première question.
Puisque les polynômes $P_n$ sont définis comme les polynômes $p_n$,
on peut utiliser cette récurrence \eqref{equ_recur}.
Soient les polynôme $p_n(x) = P_n(x)2^{-n}$, la récurrence devient
\begin{align*}
  p_0(x) & = 1\\
  p_1(x) & = \frac{x}{2}\\
  p_n(x) & = (x - 0)p_{n-1}(x) - \left(\frac{1}{2}\right)^2p_{n-2}(x).
\end{align*}

En faisant l'analogie entre les deux récurrences,
on a donc $\alpha_n = 0$ et $\beta_n = \frac{1}{2}$
(on aura pu également choisir $\beta_n = -\frac{1}{2}$,
cela n'aurait rien changé).
Puisque la relation de récurrence est valable pour $n \geq 2$, on a donc défini $\alpha_n$ et $\beta_n \forall n \geq 2$.\\ Puisqu'on doit également avoir $P_1(x) = x$, on doit imposer $\alpha_1 = 0$. Les polynômes $p_n$ sont donc définis comme suit :
%On remarque que pour satisfaire cette équation et les conditions initiales $P_0$ et $P_1$, $P_n(x)$ doit avoir la forme :
$$
p_n(x) = \det
\left[
  \begin{array}{cccccc}
    x & \frac{1}{2} & & & &  \\
    \frac{1}{2} & x & \frac{1}{2} & & & \\
      & \frac{1}{2} & x & \frac{1}{2} & & \\
      & & \ddots & \ddots & \ddots & \\
      & & & \ddots & \ddots &  \frac{1}{2}\\
      & & & &  \frac{1}{2} & x
  \end{array}
\right].
$$
Sans surprise, en calculant $p_n$ par la méthode des cofacteurs, on retrouve l'équation de récurrence de $p_n$
$$
p_n(x) = x \det
\left[
\begin{array}{ccccc}
x & \frac 12 & & &  \\
  \frac 12 & x & \frac 12 & & \\
 & \frac 12 & x & \ddots & \\
 & & \ddots & \ddots & \frac 12  \\
 & & & \frac 12 & x   \\
\end{array}
\right] - \frac 12\det
\left[
\begin{array}{cccccc}
x & \frac 12 & & & & 0 \\
\frac 12 & x & \frac 12 & & & 0\\
 & \frac 12 & x & \ddots & & 0\\
 & & \ddots & \ddots & \frac 12 & 0\\
 & & & \frac 12 & x &  0\\
 & & & &  \frac 12 &  \frac 12
\end{array}
\right] = xp_{n-1}(x)-\frac{1}{4}p_{n-2}(x).
$$

Le facteur positif $2^n$ n'a aucune importance ici,
$p_n$ et $P_n$ ont les même racines.

La matrice $T_{20}$ vaut
\[
  \begin{pmatrix}
    0 & \frac 12 & 0 & 0 & 0 & \cdots & 0 & 0\\
    \frac 12 & 0 & \frac 12 & 0 & 0 & \cdots & 0 & 0\\
    0 & \frac 12 & 0 & \frac 12 & 0 & \cdots & 0 & 0\\
    0 & 0 & \frac 12 & 0 & \frac 12 & \cdots & 0 & 0\\
    \vdots & \ddots& \ddots& \ddots& \ddots& \ddots & \vdots & \vdots\\
    0 & 0 & 0 & 0 & 0 & \cdots & 0 & \frac 12\\
    0 & 0 & 0 & 0 & 0 & \cdots & \frac 12 & 0
  \end{pmatrix}
\]
Les valeurs propres de $T_{20}$ s'obtiennent en déterminant le noyau
de l'application linéaire $\lambda I_{20} - T_{20}$ ou,
autrement dit, les racines du polynôme caractéristique donné par
$\det(\lambda I_{20} - T_{20})$.
À la question précédente, nous avons implémenté une méthode permettant
d'évaluer en $\bigoh(n)$ opérations la suite des polynômes $p_i$ en un point.
%Notons que nos polynômes $P_i$ peuvent s'écrire sous la forme :
%\begin{equation} \label{equ_tche}
%P_n(x) = 2^{n-1} ((x-\alpha_n)P_{n-1}(x) - \beta_n^2 P_{n-2}(x)) = 2^{n-1} (x P_{n-1}(x) - \beta_n^2 P_{n-2}(x))
%\end{equation}
En effet, pour déterminer le nombres de racines du polynôme caractéristique
dans $\left[ \dfrac{1}{2} , \dfrac{3}{4} \right]$,
il nous suffit d'évaluer la suite des polynômes
$ \{ p_0(x), p_1(x), ... , p_{20}(x) \}$ en $x=\dfrac{1}{2}$
et $x=\dfrac{3}{4}$ et d'appliquer le théorème de Sturm.
Le nombre de racines sera donné par l'expression :
$$N(\dfrac{1}{2} , \dfrac{3}{4}) = V(\dfrac{1}{2}) - V(\dfrac{3}{4})$$
Pour ce faire, on peut appliquer l'algorithme développé à la question 3
en utilisant la formule de récurrence de la première question.

La variation peut être calculée ainsi
\lstinputlisting{matlab/V.m}
et le nombre de racine entre $a$ et $b$ de $P_n$ ainsi
\lstinputlisting{matlab/main.m}
ce qui donne
\begin{lstlisting}
>> main(1/2, 3/4, 20);
N(5.0e-01,7.5e-01) = V(5.0e-01) - V(7.5e-01) = 7 - 5 = 2
\end{lstlisting}

On obtient ainsi $V(1/2) = 7$ et $V(3/4) = 5$.
On conclut grâce au théorème de Sturm que la matrice $T_{20}$
possède 2 valeurs propre dans l'intervalle
$\left[ \dfrac{1}{2} , \dfrac{3}{4} \right]$.
