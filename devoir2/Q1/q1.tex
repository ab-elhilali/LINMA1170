\section{Question 1}
Soit une image $\mathcal{I}$ \footnote{Matrice $n \times m$ dont les valeurs sont comprises entre 0 et 255.} qu'on souhaite flouter au moyen de deux matrices $T_R$ et $T_L$ : 
$$A=T_L \mathcal{I} T_R$$
où les matrices $T_R$ et $T_L$ sont de la forme : 
$$
 T = \frac{1}{k}
 \begin{bmatrix}
    1 & a 		& 			& \\
    a & \ddots 	& \ddots 	& \\
      & \ddots 	& \ddots 	& a \\
      & 		& a			& 1
  \end{bmatrix}.
$$
Nous séparons notre analyse en deux partie : (1) la détermination des valeurs adéquates de $k$ et (2) l'étude spectrale des matrices $T$.

\subsection{Facteur k}
Tout d'abord, il  est capital que, peu importe la valeur de $\textit{a}$ dans la matrice $\textit{T}$,
les coefficients de la matrice $\textit{A}$ gardent leur valeur comprise entre 0 et 255.
La valeur de $\frac{1}{k}$ doit donc permettre de ramener les valeurs obtenues après floutage dans une norme acceptable.

Vu la forme des matrices $T_L$ et $T_R$, les coefficients de $A$ seront composés
d'une somme de trois termes,
ces termes étant eux même la somme de 3 autres termes.

\begin{table}
$$
\begin{array}{|c|c|c|c|c|} 
\hline
\ddots & & \vdots & & \\
\hline
\cdots & x & u & l & \cdots\\
\hline
\cdots & y & v & m & \cdots \\
\hline
\cdots & z & w & n & \cdots \\
\hline
 & & \vdots & & \ddots \\
\hline
\end{array} 
$$
\caption{Pixels intérieurs à l'image.} 
\label{tabinter}
\end{table}



Soit $x,y,z, u, v, w, l, m \text{ et } n$, des coefficients de la matrice $\I$ (cf. table \ref{tabinter}). On "floute" cette matrice comme décrit plus haut tel que le pixel $v \in \mathcal{I} \Longrightarrow v' \in A$ : 
\begin{equation} \label{kGene}
  v' = \frac{a \frac{(ax+y+az)}{k}}{k} + \frac{\frac{(au+v+aw)}{k}}{k} + \frac{a \frac{(al+m+an)}{k}}{k}
\end{equation}

Notre but étant que la valeur de chaque élément quelconque $v' \in A$	soit comprise entre 0 et 255. 
Considérons le cas extrême, où tous les éléments de $\I$ valent 255 ($x=y=z=...=255$).
On peut donc réécrire l'équation $\ref{kGene}$ sous la forme :
\begin{eqnarray}
\frac{255+1020a+1020a^2}{k^2} &\leq & 255 \\
1+4a+4a^2 &\leq & k^2 \\
(2a+1)^2 &\leq & k^2
\end{eqnarray}
Pour flouter l'image, on modifie un pixel en lui ajoutant en quelque sorte, une "influence", une pondération des pixels autour de lui, on peut donc considérer $a$ et $k$ comme positifs.  
Vu la positivité de $a$ et $k$, on obtient finalement que
\begin{equation}
k \geq 2a+1
\end{equation}
Si cette inégalité est satisfaite, nous sommes assurés d'avoir des valeurs de $A$ comprises entre 0 et 255. Nous  pouvons ajouter une contrainte supplémentaire permettant de fixer une valeur $k$ : si les pixels $v$, $x$, $y$... ont la même valeur, l'idéal serait que le pixel $v'=v$. Ce qui correspond à l'égalité \footnote{Notons que cette propriété est vraie pour toutes les valeurs "intérieures" de $A$. Pour les bords de l'image (cf. tableau \ref{tabborder}), un pixel quelconque $e \in \I \Longrightarrow e' \in A$ tel que:  
$$e' = \frac{a \frac{(d+ag)}{k}}{k} + \frac{ \frac{(e+ah)}{k}}{k} + \frac{a \frac{(f+ai)}{k}}{k} . $$ On a bien trois termes de moins que dans le cas du pixel intérieur. La valeur de $e'$ sera toujours comprise entre 0 et 255 mais si $d=e=f=g=h=i$ on aura $e'= e \frac{1+3a+2a^2}{(1+2a)^2} < e$ (c'est à dire que le bord sera noirci, étant donné que 0 correspond au noir et 255 au blanc).}  $k=2a+1$. 

\begin{table}
$$
\begin{array}{|c|c|c|c|c|} 
\hline
\cdots & d & e & f & \cdots\\
\hline
\cdots & g & h & i & \cdots\\
\hline
\vdots &  & \vdots &  & \ddots \\
\hline
\end{array} 
$$
\caption{Pixels sur le bord de l'image.}
\label{tabborder}
\end{table}

\subsection{Valeurs propres}
Nous savons que les valeurs propres de la matrice
$$
S =
 \begin{bmatrix}
    0 & 1 		& 			& \\
    1 & \ddots 	& \ddots 	& \\
      & \ddots 	& \ddots 	& 1 \\
      & 		& 1			& 0
  \end{bmatrix}
$$

sont données par $\lambda_i(S) = 2 \cos(\frac{i\pi}{n+1})$, $i= 1, \ldots , n$ où $n$ est la dimension de la matrice.

Vu que $T$ est de la forme

$$
 T = \frac{1}{k}
 \begin{bmatrix}
    1 & a 		& 			& \\
    a & \ddots 	& \ddots 	& \\
      & \ddots 	& \ddots 	& a \\
      & 		& a			& 1
  \end{bmatrix},
$$

on a donc que $T = \frac{1}{k} I + \frac{a}{k} S$. Ceci implique également que tout vecteur propre de $T$ est également vecteur propre de $S$ et inversément.

Soit $v$ un vecteur propre de $S$, on a que :
\begin{equation}
T v = \left( \frac{1}{k} I + \frac{a}{k} S \right) v =
\frac{1}{k} v + \frac{a}{k} \lambda(S) v = \left(\frac{1}{k} + \frac{a}{k} \lambda(S)\right) v
\end{equation}
Les valeurs propres de $T$ sont donc du type $\lambda_i(T) = \frac{1}{k} + 2 \frac{a}{k} \cos(\frac{i\pi}{n+1})$.

On considère une matrice de très grande taille et donc un $n$ qui tend vers l'infini\footnote{ Pour l'image donnée, en réalisant cette approximation, on obtient un $\kappa^2 $ pour $a = 0.4$ (cf. suite du rapport) de  81 au lieu de 80.862, comme il s'agit d'une borne supérieur sur un accroissement de perturbations proportionnellement très petites c'est approximation fait sens.}. 
On obtient donc comme valeurs propres maximales $\lmax$ et minimales $\lmin$,
pour $a$ et $k$ positifs :

$$\lmax(T) \underset{n\rightarrow \infty}{\longrightarrow} \frac{1+2a}{k} $$
$$\lmin (T) \underset{n\rightarrow \infty}{\longrightarrow} \frac{1-2a}{k} $$

Le nombre de conditionnement
\footnote{On dit d'une matrice qu'elle est bien conditionnée si son nombre de conditionnement
  est proche de 1 et mal conditionnée s'il est grand.
En effet, dans le cadre d'un problème de type $Ax = b$,
la valeur de ce nombre fournit une borne de l'erreur relative de $x$ quand on introduit une perturbation $\Delta A $ ou $\Delta b$.
Dans notre cas on cherchera donc à prendre un $a$ le plus proche possible de 0.
\textit{http://fr.wikipedia.org/wiki/Conditionnement$\_$(analyse$\_$numérique)}}
 $\kappa$ de la matrice $T$ (symétrique) est défini par
\begin{equation}
	\kappa (T) = ||T||_2||T^{-1}||_2 = \frac{\lmax (T)}{\lmin (T)} \underset{n\rightarrow \infty}{\longrightarrow} \frac{1+2a}{1-2a}
\end{equation}

La norme subordonnée à la norme $||T||_2$ est définie telle que :
$$||T||_2 = \sqrt{\lmax(T^TT)}$$
où $\lmax(T^TT)$ désigne la plus grande valeur propre de $T^TT$.
On sait donc que $\kappa(T) $ est positif.
De plus, comme $\lmax(T)$ est forcément plus grand ou également à $\lmin$, on a que $\kappa(T) \geq 1$.
On remarque également que pour $a \geq 0.5$ le nombre de conditionnement n'est pas défini.
C'est le cas de manière évidente pour $a = 0.5$ et c'est également le cas pour $a > 0.5$ parce que $\kappa$ doit être positif.

Le carré du nombre de conditionnement est :
\begin{equation} \label{equatkappa}
	\kappa^2 (T) = \left(\frac{1+2a}{1-2a}\right)^2
\end{equation}

Celui-ci donne une idée de l'accroissement possible de $A$ par rapport à $\I$.
En effet, vu que la transformation est du type $A = T_L \I T_R$,
le conditionnement de $T_L$ et de $T_R$ interviennent,
il faut donc considérer le carré du nombre de conditionnement pour avoir une idée de l'accroissement de $A$. On constate que pour une valeur de $a$ de $0.4$, on obtient $\kappa^2 (T) = 81$, qui est déjà une valeur fort conséquente quant à l'amplification d'une perturbation sur la matrice initiale.
