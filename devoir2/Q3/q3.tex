\section{Question 3}

\subsection{Jacobi}
% TODO comparer itérations
On remarque tout d'abord à l'aide de la figure~\ref{fig:arj}
et de la figure~\ref{fig:rhoj} que pour $a \geq 0.25$,
Jacobi ne converge pas pour tout $r$ mais uniquement à partir
d'un certain $r$.

Le $r$ pour lequel Jacobi commence à converge correspond d'ailleurs
assez bien au point où le rayon spectral passe en dessous de 1.

La figure~\ref{fig:rhoj} nous montre que le rayon spectral est
décroisant en fonction de $r$ ce qui nous invite à penser que
la convergence sera de plus en plus rapide mais ça ne nous
dit rien sur l'évolution de l'erreur.

La figure~\ref{fig:arj} nous donne un aperçu de l'impact de $r$
sur l'erreur pour différentes valeurs de $a$.
On voit qu'en choisissant correctement $r$ on a une erreur semblable
pour $a \leq 0.25$ mais pour $a \geq 0.3$, on a une nette
diminution de l'erreur.
Pour $a = 0.45$ (pas représenta sur la figure~\ref{fig:rhoj}),
l'erreur était de 7000 mais avec la régularisation, on obtenait
1500.

On voit que pour $a \geq 0.3$, la régularisation donne de meilleurs résultats
lorsqu'on choisit bien $r$.
Pour $a \leq 0.25$ par contre, la régularisation donne de moins bons résultats.

Le meilleur $r$ pour $a \geq 0.3$ semble être le premier $r$ à partir duquel
ça converge.
Pour $a \leq 0.25$, ça n'a pas l'air aussi simple mais la fonction semble
unimodale donc on peut aisément trouver le $r$ qui minimise l'erreur.

Une recherche binaire trouvant le $r$ pour lequel la méthode commence à converger
suivit d'une recherche unimodale nous permettra donc de voir le $r$ optimal en
fonction de $a$.

\subsection{Gauss-Seidel}
La figure~\ref{fig:rhogs} nous invite à penser que pour
$a < 0.5$, le rayon spectral est plus petit que 1.
On voit d'ailleurs avec la figure~\ref{fig:args} que contrairement
à Jacobi, on a pas de problème de convergence.

On a aussi une erreur semblable pour un $r$ bien choisi
avec et sans la régularisation pour $a \leq 0.25$ et une nette
diminution pour $a \geq 0.3$.

La fonction de l'erreur en fonction de $r$ semble à nouveau unimodale
ce qui nous permettra d'effectuer une recherche unimodale.

\begin{figure}
  \centering
  \begin{subfigure}[b]{0.45\textwidth}
    \includegraphics[width=\textwidth]{Q3/arJacobi.png}
    \caption{Erreur pour différents $a$ en fonction de $r$ de Jacobi.
      Lorsque la méthode ne converge pas, le point n'est pas représenté.
      L'erreur est calculée comme la norme de Frobenius entre l'image de départ
    et l'image défloutée.}
    \label{fig:arj}
  \end{subfigure}%
  ~
  \begin{subfigure}[b]{0.45\textwidth}
    \includegraphics[width=\textwidth]{Q3/arGauss-Seidel.png}
    \caption{Erreur pour différents $a$ en fonction de $r$ de Gauss-Seidel.
      Lorsque la méthode ne converge pas, le point n'est pas représenté.
      L'erreur est calculée comme la norme de Frobenius entre l'image de départ
    et l'image défloutée.}
    \label{fig:args}
  \end{subfigure}

  \begin{subfigure}[b]{0.45\textwidth}
    \includegraphics[width=\textwidth]{Q3/rhoJacobi.png}
    \caption{$\rho(M^{-1}N)$ pour différents $a$ en fonction de $r$ de Gauss-Seidel
    calculé numériquement.}
    \label{fig:rhoj}
  \end{subfigure}%
  ~
  \begin{subfigure}[b]{0.45\textwidth}
    \includegraphics[width=\textwidth]{Q3/rhoGauss-Seidel.png}
    \caption{$\rho(M^{-1}N)$ pour différents $a$ en fonction de $r$ de Jacobi
    calculé numériquement.}
    \label{fig:rhogs}
  \end{subfigure}
\end{figure}
