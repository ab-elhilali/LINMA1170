\section{Question 3}

\subsection{Jacobi}
% TODO comparer itérations
La figure~\ref{fig:ar} nous donne un aperçu de l'impact de $r$
sur l'erreur pour différentes valeurs de $a$.

On voit que pour $a \geq 0.3$, la régularisation donne de meilleurs résultats
pour autant qu'on choisissse un bon $r$.
Pour $a \leq 0.25$ par contre, la régularisation donne de moins bons résultats.

On a d'ailleurs pas toujours la convergence pour $a \geq 0.25$ mais à partir
d'un certain $r$, ça converge.
Le meilleur $r$ pour $a \geq 0.3$ semble être le premier $r$ à partir duquel
ça converge.
Pour $a \leq 0.25$, ça n'a pas l'air aussi simple mais la fonction semble
unimodale donc on peut aisément trouver le $r$ qui minimise l'erreur.

Une recherche binaire trouvant le $r$ pour lequel la méthode commence à converger
suivit d'une recherche unimodale nous permettra donc de voir le $r$ optimal en
fonction de $a$.

\begin{figure}
  \centering
  \includegraphics[width=\textwidth]{Q3/arJacobi.png}
  \caption{Erreur pour différents $a$ en fonction de $r$ de Jacobi.
  Lorsque la méthode ne converge pas, le point n'est pas représenté.
  L'erreur est calculée comme la norme de Frobenius entre l'image de départ
  et l'image défloutée.}
  \label{fig:ar}
\end{figure}

\subsection{Gauss-Seidel}

\begin{figure}
  \centering
  \includegraphics[width=\textwidth]{Q3/arGauss-Seidel.png}
  \caption{Erreur pour différents $a$ en fonction de $r$ de Gauss-Seidel.
  Lorsque la méthode ne converge pas, le point n'est pas représenté.
  L'erreur est calculée comme la norme de Frobenius entre l'image de départ
  et l'image défloutée.}
  \label{fig:ar}
\end{figure}

\begin{table}
  \centering
  \begin{tabular}{|l|l|l|l|l|}
    \hline
    \multirow{2}{*}{$a$} & \multicolumn{2}{l|}{Jacobi} & \multicolumn{2}{l|}{Gauss-Seidel}\\
    \cline{2-5}
        & $i_1$ & $i_2$ & $i_1$ & $i_2$\\
    \hline
    0.2 &     &     &     & \\
    \hline
    0.4 & 71    &     &     & \\
    \hlin
    0.45&   &    &     & \\
    \hline
    0.5 &    &   &    & \\
    \hline
  \end{tabular}
  \caption{Nombre d'itération nécessaire pour déflouter pour différentes valeurs de $a$ (le défloutage abandonne à 500 itérations).
  $i_1$ est le nombre d'itérations pour le premier système et $i_2$ pour le deuxième.}
  \label{tab:iter}
\end{table}

\subsection{Analyse du rayon spectral}


\begin{figure}
  \centering
  \includegraphics[width=\textwidth]{Q3/rhoJacobi.png}
  \includegraphics[width=\textwidth]{Q3/rhoGauss-Seidel.png}
  \caption{Les valeurs de \lambda pour différentes valeurs de a et de r}
  \label{fig:ar}
\end{figure}
