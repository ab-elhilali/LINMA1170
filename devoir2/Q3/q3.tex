\section{Question 3}

\subsection{Jacobi}
% TODO comparer itérations
La figure~\ref{fig:ar} nous donne un aperçu de l'impact de $r$
sur l'erreur pour différentes valeurs de $a$.

On voit que pour $a \geq 0.3$, la régularisation donne de meilleurs résultats
lorsqu'on choisit bien $r$.
Pour $a \leq 0.25$ par contre, la régularisation donne de moins bons résultats.

On a d'ailleurs pas toujours la convergence pour $a \geq 0.25$ mais à partir
d'un certain $r$, ça converge.
Le meilleur $r$ pour $a \geq 0.3$ semble être le premier $r$ à partir duquel
ça converge.
Pour $a \leq 0.25$, ça n'a pas l'air aussi simple mais la fonction semble
unimodale donc on peut aisément trouver le $r$ qui minimise l'erreur.

Une recherche binaire trouvant le $r$ pour lequel la méthode commence à converger
suivit d'une recherche unimodale nous permettra donc de voir le $r$ optimal en
fonction de $a$.

\begin{figure}
  \centering
  \includegraphics[width=\textwidth]{Q3/arJacobi.png}
  \caption{Erreur pour différents $a$ en fonction de $r$ de Jacobi.
  Lorsque la méthode ne converge pas, le point n'est pas représenté.
  L'erreur est calculée comme la norme de Frobenius entre l'image de départ
  et l'image défloutée.}
  \label{fig:ar}
\end{figure}

\subsection{Gauss-Seidel}

\begin{figure}
  \centering
  \includegraphics[width=\textwidth]{Q3/arGauss-Seidel.png}
  \caption{Erreur pour différents $a$ en fonction de $r$ de Gauss-Seidel.
  Lorsque la méthode ne converge pas, le point n'est pas représenté.
  L'erreur est calculée comme la norme de Frobenius entre l'image de départ
  et l'image défloutée.}
  \label{fig:ar}
\end{figure}
