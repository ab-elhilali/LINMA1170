\section{Question 2}
\subsection{Comparaison Jacobi et Gauss-Seidel}
Voici un tableau qui dresse un bilan comparatif des méthodes de Jacobi et de Gauss-Seidel pour différentes valeurs de $k$ et de $a$.
\footnote{Les codes des méthodes de Jacobi et Gauss-Seidel définissent une solution $x$
tel que $|x-x^*|<\epsilon$ où $x^*$ est la solution exacte et $\epsilon$ a été arbitrairement fixé à $0.00001$}

\begin{tabular}{|c|c|c|}
\hline
& nombres d'itérations & temps $[s]$\\
\hline
Jacobi ($k=0.02$, $a=0.4$) &98 &  0.824813\\
Gauss-Seidel ($k=0.02$, $a=0.4$) &43 & 0.379432 \\
\hline
Jacobi ($k=0.02$, $a=0.45$) &206 & 1.639770\\
Gauss-Seidel ($k=0.02$, $a=0.45$) & 79 & 0.656296 \\
\hline
Jacobi ($k=0.02$, $a=0.5$) & $\infty$ & $\infty$\\
Gauss-Seidel ($k=0.02$, $a=0.5$) & $\infty$ & $\infty$\\
\hline
Jacobi ($k=$, $a=$) & & \\
Gauss-Seidel ($k=$, $a=$) & & \\
\hline
\end{tabular}

\subsection{Rayons spectraux}
Le problème de départ $T_L X T_R = A_{\Delta}$ peut se décomposer en deux étapes \footnote{Si $A_{\Delta}$
est une matrice de dimension $n*m$, $T_L$ (resp. $T_R$) sera une  matrice carré de dimension $n*n$ (resp. $m*m$).} :
\begin{eqnarray}\label{eq_q2}
T_L \mathcal{Y} &=& A_{\Delta}\\
T_R \mathcal{X}^T &=& \mathcal{Y}^T.
\end{eqnarray}

Les matrices de Jacobi et de Gauss-Seidel sont définies comme suit :
\begin{eqnarray}
\mathcal{J} &=& -D^{-1}(L+U)\\
\mathcal{G} &=& -(D+L)^{-1}U,
\end{eqnarray}
autrement dit :
\begin{eqnarray}
\mathcal{J} &=& - a
\left[
\begin{array}{cccc}
1 & & &\\
 & 1 & &\\
 & & \ddots & \\
  & & & 1
\end{array}
\right]
\left[
\begin{array}{cccc}
0 & 1& &\\
1 & \ddots & \ddots &\\
 & \ddots & \ddots & 1 \\
  & & 1 & 0
\end{array}
\right]\\
\mathcal{G} & = & -
\left[
\begin{array}{cccc}
1 & & &\\
a & \ddots &  &\\
 & \ddots & \ddots  &  \\
  & & a & 1
\end{array}
\right] ^{-1}
\left[
\begin{array}{cccc}
0 & a & &\\
 & \ddots &  \ddots &\\
 &  & \ddots  & a \\
  & &  & 0
\end{array}
\right]
\\
&=& -
\left[
\begin{array}{ccccc}
1 & & & &\\
-a & \ddots &  & &\\
a^2 & \ddots & \ddots  &  &\\
 \vdots & \ddots & \ddots & \ddots & \\
(-1)^{n-1}a^{n-1} & & a^2 & -a & 1
\end{array}
\right]
\left[
\begin{array}{cccc}
0 & a & &\\
 & \ddots &  \ddots &\\
 &  & \ddots  & a \\
  & &  & 0
\end{array}
\right]\\
&=& -
\left[
\begin{array}{ccccc}
0 & a & & &\\
0 & -a^2 & \ddots  & &\\
\vdots & a^3 & \ddots & \ddots  &\\
\vdots & \vdots & \ddots & \ddots  & a \\
0 & (-1)^n a^n & & a^3 & -a^2
\end{array}
\right]
\end{eqnarray}

Pour la méthode de Jacobi, on obtient directement (cf. question 1) que les valeurs propres de $\mathcal{J}$ valent
$\lambda _i = -2a * cos(\dfrac{i \pi}{n+1})$ et donc $\rho(\mathcal{J}) = max |\lambda _i| < 2a$.
Ainsi la méthode de Jacobi convergera pour des valeurs de $a \leq 0.5$.

\subsection{Représentation des résultats en fonction de $a$}
