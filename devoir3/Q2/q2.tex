\section*{Question 2}

\subsection*{Newton-Raphson}

Utilisons le théorème $3.10$ énoncé à la question 1.
Dans notre cas, on voit aisément que $J_G(y) = W_{\bot}^*AW_{\bot} - w^*AwI - (yw^*AW_{\bot}+Iw^*AW_{\bot}y) $ est continu en $s$ ($I$ est la matrice identité de taille $n-1$).

On peut étendre le théorème $3.10$ avec des hypothèses plus forte sur la jacobienne : 

Si, de plus, il existe une constante $\alpha > 0$ telle que la condition de Lipschitz 
$$||DF(x) - DF(s)|| \leq \alpha || x - s||, \forall x \in \Omega,$$
alors l'ordre de convergence est au moins 2. 

Dans notre cas, en utilisant l'inégalité de Cauchy et l'inégalité triangulaire, on obtient que : 
\begin{eqnarray}
||DG(x) - DG(s)|| &=& ||W_{\perp}^{*} A W_{\perp}- w^{*} A wI - (xw^*AW_{\bot}+Iw^*AW_{\bot}x) - W_{\perp}^{*} A W_{\perp}+ w^{*} A wI + (sw^*AW_{\bot}+Iw^*AW_{\bot}s) || \\
||DG(x) - DG(s)|| &=&  ||(sw^*AW_{\bot}+Iw^*AW_{\bot}s) - (xw^*AW_{\bot}+Iw^*AW_{\bot}x) || \\
 ||DG(x) - DG(s)|| &\leq & (||w^*AW_{\bot}|| + || Iw^*AW_{\bot} ||) || x-s ||
\end{eqnarray}
Par identification on a $\alpha = ||w^*AW_{\bot}|| + || Iw^*AW_{\bot} || \geq 0$, et donc la condition de Lipschitz sur $DG(x)$ est satisfaite.On conclut qu'on a bien un ordre de convergence d'au moins 2 pour la méthode de Newton appliquée à la fonction $G$.

\subsection*{Rayleigh symétrique}

\textbf{Proposition 4.4} Soit $A = A^*$ $ \in \mathbb{C}^{n\times n}$. L'itération du quotient de Rayleigh pour $A$ converge vers une direction propre de $A$ pour presque tout itéré initial. Lorsque la suite des itérés converge vers une direction propre, la convergence est cubique. 

Il faut maintenant préciser ce \textit{pour presque tout itéré initial}. On ne peut pas choisir n'importe quel itéré initial. En effet, soit $x_0 = a_1 v_1 + \ldots + a_n v_n $ la décomposition dans la base ($v_1$, \ldots , $v_n$ ) de $\mathbb{C}^n$ formée des vecteurs propres de $A$. En considérant que $v_1$ est le vecteur propre pour la valeur propre $\lambda_1$ dominante, il faut que $a_1$ soit non nul dans l'itéré initial pour avoir convergence, sinon le vecteur initial n'aura aucune composante dans la bonne direction.   

\subsection*{Rayleigh asymétrique}