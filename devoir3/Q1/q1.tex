\section*{Question 1}
Soit la fonction vectorielle suivante : $$F : x \neq 0 \rightarrow Ax - \frac{x^{*}Ax}{x^*x}x.$$
On a que $x \neq 0$ est un vecteur propre de $A$ si et seulement si $F(x)=0$. A la place de rechercher les valeurs propres de $A$, on peut donc chercher les zéros de la fonction $F$. Puisque cette fonction est une fonction vectorielle, elle est de la forme : 
$$F : 
\left(\begin{array}{c}
x_1 \\
\vdots \\
x_n
\end{array}\right)
\longmapsto
\left(\begin{array}{c}
f_1(x_1,...,x_n) \\
\vdots \\
f_n(x_1,...,x_n)
\end{array}\right)
$$
où les fonctions $f_1,...,f_n$ sont des fonctions scalaires.\\
Montrons tout d'abord que $F(\alpha x) = \alpha F(x)$ pour tout $\alpha$ scalaire non nul.\\
Pour se faire, nous allons simplement utiliser la définition de la fonction $F$. On a $$F(\alpha x) = A(\alpha x) - \frac{(\alpha x)^*A(\alpha x)}{(\alpha x)^*(\alpha x)}\alpha x = \alpha Ax - \frac{\alpha^2 x^* A x}{\alpha^2 x^* x}\alpha x = \alpha F(x)$$

Nous allons maintenant utiliser cette propriété pour montrer que la jacobienne de la fonction $F$ est singulière lorsque $F$ s'annule. Autrement dit, il faut montrer que $$F(s) = 0 \Rightarrow J_{F}(s) \text{ singulière}.$$
Pour rappel, la jacobienne d'un fonction vectorielle $F$ telle que celle ci-dessus est donnée par 
$$J_F(x) =
\left(\begin{array}{c}
(\nabla f_1(x))^T \\
\vdots \\
(\nabla f_n(x))^T
\end{array}\right).$$
Utilisons le développement en série de Taylor de la fonction $F$ autour de $s$.\\
$$\forall \delta \geq 0, \exists \theta \in [0;\delta], F(s+\delta s) = F(s) + \delta J_F(s+\theta s)s$$
A l'aide de la propriété montrée ci-dessus, on a la relation suivante  $$\forall \delta \geq 0, \exists \theta \in [0;\delta], (1+\delta)F(s) = F(s) + \delta J_F(s+\theta s)s$$
On utilise maintenant le fait que $F(s) = 0$ et on trouve $$\forall \delta \geq 0, \exists \theta \in [0;\delta], \delta J_F(s+\theta s)s = 0$$
Etant donné que cette relation est vraie $\forall \delta \geq 0$, elle est aussi valable $\forall \delta > 0$. Le terme $\delta$ pouvant valoir n'importe quel réel strictement positif, on peut diviser par $\delta$ à gauche et à droite de l'expression ci-dessus. On a alors la relation suivante : $$\forall \delta > 0, \exists \theta \in [0;\delta], J_F(s+\theta s)s = 0$$
Par continuité de la jacobienne, on déduit que $$J_F(s)s = 0$$
%La dérivée d'une fonction scalaire à une variable est définie comme suit : $$g'(x) = \lim_{h \to 0} \frac{g(x+h)-g(x)}{h}$$
%Nous allons essayer d'adapter cette définition pour l'appliquer à la jacobienne de la fonction $F$. Nous allons d'abord l'adapter aux fonctions scalaires à plusieurs variables $f_i$ qui composent $F$. On peut montrer qu'on a $$\lim_{h \to 0} (\nabla f_i(x))^T h = \lim_{h \to 0} f_i(x+h)-f_i(x), i = 1,...,n$$ où $x$, $h$ et $\nabla f_i(x)$ sont des vecteurs colonnes de même tailles. %Ca serait cool de le montrer un peu mieux
%Puisque la jacobienne d'une fonction vectorielle telle que $F$ est définie comme 
%$$J_F(x) =
%\left(\begin{array}{c}
%(\nabla f_1(x))^T \\
%\vdots \\
%(\nabla f_n(x))^T
%\end{array}\right),$$
%on peut adapter la définition du gradient pour des fonctions scalaires à une définition du jacobien pour des fonctions vectorielles. On a alors le résultat suivant\footnote{Une autre approche assez similaire aurait été d'évaluer le développement en série de Taylor de la fonction $F$ autour de $s$ et de l'évaluer en $s + \delta s$.} : $$\lim_{h \to 0} J_F(x) h = \lim_{h \to 0} F(x+h) - F(x).$$
%Afin d'utiliser la propriété montrée ci-dessus, on peut réécrire cette définition de la jacobienne comme suit : $$\lim_{\delta \to 0} J_F(x) \delta x = \lim_{\delta \to 0} F(x+\delta x) - F(x) = \lim_{\delta \to 0} \delta F(x)$$ où on peut diviser à gauche et à droite par $\delta$ et on a donc plus besoin de passer par la limite. Dans le cas $x=s$ tel que $F(s)=0$, on a alors $J_F(s)s=0$. 
C'est-à-dire que $s$ est vecteur propre de $J_F(s)$, associé à la valeur propre $0$. Ceci implique que $0$ est solution de l'équation $\text{det}(J_F(s)-xI)=0$.\\
On a donc $$\text{det}(J_F(s))=0$$ et ceci ne peut-être vrai que si $J_F(s)$ est une matrice singulière.

Nous sommes ici dans le cas de la méthode de Newton multidimensionnelle et le seul résultat du cours donnant des informations sur la convergence de cette méthode est le théorème 3.10. Enonçons-le.\\
\textbf{Théorème 3.10.} Soit $F : \mathbb{R}^n \rightarrow \mathbb{R}^n$ une fonction dérivable dans un voisinage ouvert $\Omega$ d'un point $s$ pour lequel $F(s)=0$. Supposons que $J_F$ est continue en $s$ et que $J_F(s)$ est non singulière. Alors $s$ est un point d'attraction de l'itération de Newton $$\mathcal{I} : x_{k+1} = x_k - J_F(x_k)^{-1}F(x_k), k = 0,1,...,$$ et la convergence est superlinéaire en $s$.\\
Puisque, comme nous l'avons montré ci-dessus, la jacobienne $J_F$ est singulière en $s$, ce théorème n'est pas applicable. Par conséquent, on ne peut conclure quoique ce soit sur la convergence locale ou globale de la méthode.

%La méthode de Newton définie la fonction d'itération $\mathcal{I}$ tel que : $$f:\mathbb{R^n} \Rightarrow \mathbb{R^n} : x \rightarrow F(x)-[DF^(x)]^{-1}F(x). $$
%On a bien $\forall s | F(s)=0 \Longrightarrow f(s)=0$ si $DF$ est régulière. Ainsi s est un point fixe de la fonction $f$. Par le théorème 3.10 du cours, ce point fixe est attractif et la convergence est superlinéaire. Or comme ici la jacobienne $DF(x)$ est singulière en $s$, aucun résultats ne nous permet de conclure quoique ce soit sur sa convergence locale ou globale.


Montrons à présent le point 4. On définit $w \in \mathbb{C}_0^n$ de norme 1 et $W_{\bot}$, une base orthonormée du complément orthogonal de $w$. La matrice $W_{\bot}$ est une matrice composée des vecteurs de base de l'espace associé. Elle est, par conséquent, de taille $n \times (n-1)$ et est de la forme $$W_{\bot} = (v_1 v_2 \hdots v_{n-1})$$ où les $v_i$ sont des vecteurs colonnes de taille $n$. Puisque les $v_i$ sont les vecteurs d'une base orthonormée, on a $v_i^*v_j = \delta_{ij}$ et, puisque cette base est la base d'un espace orthogonal à $w$, on a également $v_i^*w = 0$ et ce $\forall i = 1,...,n-1$.%, c'est-à-dire une matrice dont les colonnes sont les vecteurs de la base de l'espace $ \lbrace v\in \mathbb{C^n_0} | (v|w)=0 \rbrace$. $W_{\bot}$ est bien une matrice de taille $n \times (n-1)$. On considère la fonction : 
On cherche à montrer que la fonction
\begin{equation} \label{eq_point4_q1}
 G:y \rightarrow W_{\bot}^* A(w+ W_{\bot}y) - w^*A(w+ W_{\bot}y)y,
\end{equation}
vaut $0$ lorsqu'elle est évaluée en $y$ si $(w+ W_{\bot}y)$ est un vecteur propre de $A$. La relation à montrer est donc la suivante : $$A(w+ W_{\bot}y) = \lambda (w+ W_{\bot}y) \Rightarrow G(y)=0.$$

Pour rester cohérent au niveau des dimensions des objets manipulés, le vecteur colonne $y$ doit être de taille $n-1$. C'est en effet le cas puisque le vecteur $y$ appartient à l'espace engendré par $W_{\bot}$ et ses entrées correspondent à ses coordonnées dans la base $W_{\bot}$. De façon plus explicite, on a que le vecteur $y$ est de la forme $y = a_1 v_1 + a_2 v_2 + ... + a_{n-1} v_{n-1}$ lorsqu'il est exprimé dans la base de l'espace total de taille $n$ et il est de la forme $y = (a_1 a_2 ... a_{n-1})^{T}$ lorsqu'il est exprimé dans la base $W_{\bot}$. On remarque assez facilement que la matrice $W_{\bot}$ permet de passer d'une base à l'autre. En effet, on a $$a_1 v_1 + a_2 v_2 + ... + a_{n-1} v_{n-1} = W_{\bot}
\left(\begin{array}{c}
a_1 \\
\vdots \\
a_{n-1}
\end{array}\right)
\text{et}
\left(\begin{array}{c}
a_1 \\
\vdots \\
a_{n-1}
\end{array}\right)
= W_{\bot}^*(a_1 v_1 + a_2 v_2 + ... + a_{n-1} v_{n-1})$$ où $W_{\bot}^*$ est la matrice adjointe \footnote{Une matrice adjointe d'une matrice $M$ à coefficients complexes est la matrice transposée de la matrice conjuguée de $M$ (qui est la matrice formée des éléments de $M$ conjugués).} de $W_{\bot}$.
%Où $y$ est donc un vecteur de taille $n-1$, coordonnées dans la base $W_{bot}$; $M^*$ est la matrice adjointe \footnote{Une matrice adjointe d'une matrice $M$ à coefficients complexes est la matrice transposé de la matrice conjuguée de $M$ (qui est la matrice formée des éléments de $M$ conjugués).} de $M$.
Montrons que si $(w+ W_{\bot}y)$ est un vecteur propre de $A$, alors $G(y)=0$.
\begin{proof}
Soit $(w+ W_{\bot}y) \in Ker(A-\lambda I)$. L'équation \ref{eq_point4_q1} se réécrit alors : 
\begin{eqnarray}
G(y) &=& W_{\bot}^* \lambda (w+ W_{\bot}y) - w^*\lambda (w+ W_{\bot}y)y\\
G(y) &=& \lambda [W_{\bot}^* w + W_{\bot}^*W_{\bot}y - w^* wy + w^* W_{\bot}yy ] 
\end{eqnarray}
Puisque la matrice $W_{\bot}$ est composée de vecteurs colonnes orthogonaux à $w$, on remarque assez aisément que $W_{\bot}^*w = 0$. On peut également mettre en évidence le fait que $W_{\bot}^* W_{\bot} = I$ lorsque cet opérateur est appliqué à un vecteur appartenant à l'espace orthogonal à $w$. Cela fait de la matrice $W_{\bot}$ une matrice orthogonal. De plus, comme $w$ est de norme unitaire, on a $w^* w=1$. Finalement, on observe que $w^*W_{\bot}y = 0$ car $W_{\bot}y$ est le vecteur $y$ exprimé dans les coordonnées de l'espace total, ce vecteur étant forcément orthogonal à $w$. Sachant tout cela, on peut réécrire l'équation ci-dessus comme ceci :
%Comme $W_{\bot}$ est orthogonal, on a $W_{\bot}^* W_{\bot} = I$. De plus, comme $w$ est de norme 1, $w^* w=1$. Enfin par définition de $W_{bot}$: $(w| W_{\bot}y)=0$. La dernière équation se réécrit alors : 
$$G(y) = \lambda [0 + Iy - y + 0] = 0 $$
\end{proof}

Calculons maintenant la jacobienne $J_G$ de cette fonction vectorielle. Pour commencer, nous allons calculer les jacobiennes de quelques fonctions type. Ces jacobiennes vont ensuite nous être utiles pour calculer la jacobienne de la fonction $G$.\\
Soit la fonction $f(x) = \alpha x$ où $x$ est un vecteur colonne de taille $n$ et $\alpha$ est un scalaire. Calculons la jacobienne de la fonction $f$. On trouve directement que $$J_f(x) = \alpha I.$$
Intéressons nous maintenant à la fonction $g(x) = Mx$ où $x$ est un vecteur colonne de taille $n$ et $M$ est une matrice carrée de taille $n \times n$. Calculons sa jacobienne. On trouve assez facilement que $$J_g(x) = M.$$
Passons maintenant à un dernier tpye de fonction vectorielle, les fonctions de la forme $h(x) = v^* x x$ où $x$ et $v$ sont des vecteurs colonnes de taille $n$. On a que 
$$J_h(x) =
\left(\begin{array}{cccc}
\sum_{i=1}^{n}{v_i^* x_i} + v_1^* x_1 & v_2^* x_1 & \hdots & v_n^* x_1 \\
\vdots & \vdots & \vdots & \vdots\\
v_1^* x_n & v_2^* x_n & \hdots & \sum_{i=1}^{n}{v_i^* x_i} + v_n^* x_n \\
\end{array}\right).$$
$$J_h(x) =
\left(\begin{array}{cccc}
v_1^* x_1 & v_2^* x_1 & \hdots & v_n^* x_1 \\
\vdots & \vdots & \vdots & \vdots\\
v_1^* x_n & v_2^* x_n & \hdots & v_n^* x_n \\
\end{array}\right) + I\sum_{i=1}^{n}{v_i^* x_i} = xv^*+Iv^*x.
$$
On a maintenant tous les outils dont on a besoin pour trouver la jacobienne de $G$.\\
On trouve que $$J_G(y) = W_{\bot}^*AW_{\bot} - w^*AwI - (yw^*AW_{\bot}+Iw^*AW_{\bot}y).$$
%Et finalement, attaquons nous au point 5. C'est super simple : 
%$$J_G(y) = W^*_{\bot}AW_{\bot} - w^*A(w + 2W_{\bot}) $$
%TADAM!! 

















 