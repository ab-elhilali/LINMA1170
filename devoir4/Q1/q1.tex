\section*{Question 1}
On sait que $\dot{\Ha}(p,q) = 0$ car $\Ha$ est invariant dans le temps.
De plus, comme
\[
  \dot{\Ha}(p,q) = \fpart{\Ha(p,q)}{p}^T\dot{p} + \fpart{\Ha(p,q)}{q}^T\dot{q}
\]
on doit avoir
\[
  \fpart{\Ha(p,q)}{p}^T\dot{p} + \fpart{\Ha(p,q)}{q}^T\dot{q} = 0.
\]

Les équations
\begin{align*}
  \dot{q} & = \fpart{\Ha(p,q)}{p} & \dot{p} & = \fpart{\Ha(p,q)}{q}
\end{align*}
donnent
\begin{align*}
  \fpart{\Ha(p,q)}{p}^T\dot{p} + \fpart{\Ha(p,q)}{q}^T\dot{q} & =
  \fpart{\Ha(p,q)}{p}^T\fpart{\Ha(p,q)}{q} - \fpart{\Ha(p,q)}{q}^T\fpart{\Ha(p,q)}{p}\\
  & = 0
\end{align*}
ce qui respecte bien l'invariance temporelle de $\Ha$.

Soit $f_1(q,p)$ et $f_2(q,p)$ tels que
\begin{align*}
  \dot{q} & = f_1(q,p)\\
  \dot{q} & = f_2(q,p),
\end{align*}
on calcule
\begin{align*}
  f_1(q,p) & =
  \begin{pmatrix}
    \fpart{}{p_1} \left(\frac{1}{2}(p_1^2 + p_2^2) + \frac{1}{\sqrt{q_1^2 + q_2^2}}\right)\\
    \fpart{}{p_2} \left(\frac{1}{2}(p_1^2 + p_2^2) + \frac{1}{\sqrt{q_1^2 + q_2^2}}\right)
  \end{pmatrix}\\
  & =
  \begin{pmatrix}
    p_1\\
    p_2
  \end{pmatrix}\\
  & = p\\
%
  f_2(q,p) & =
  -\begin{pmatrix}
    \fpart{}{q_1} \left(\frac{1}{2}(p_1^2 + p_2^2) + \frac{1}{\sqrt{q_1^2 + q_2^2}}\right)\\
    \fpart{}{q_2} \left(\frac{1}{2}(p_1^2 + p_2^2) + \frac{1}{\sqrt{q_1^2 + q_2^2}}\right)
  \end{pmatrix}\\
  & =
  \frac{1}{2(q_1^2 + q_2^2)^{3/2}}
  \begin{pmatrix}
    2q_1\\
    2q_2
  \end{pmatrix}\\
  & = \frac{q}{\|q\|_2^3}.
\end{align*}

On remarque que $f_1(q,p) = f_1(p)$ et $f_2(q,p) = f_2(q)$.
On va utiliser cette propriété qui va nous être particulièrement utile pour euler symplectique.
Pour euler implicite, on aura besoin de
\begin{align*}
  \fpart{f_1(p)}{p} & = I\\
  \fpart{f_2(q)}{q} & =
  \begin{pmatrix}
    \fpart{}{q_1}\frac{q_1}{(q_1^2 + q_2^2)^{3/2}} &
    \fpart{}{q_2}\frac{q_1}{(q_1^2 + q_2^2)^{3/2}}\\
    \fpart{}{q_1}\frac{q_2}{(q_1^2 + q_2^2)^{3/2}} &
    \fpart{}{q_2}\frac{q_2}{(q_1^2 + q_2^2)^{3/2}}
  \end{pmatrix}\\
  & =
  \begin{pmatrix}
    \frac{(q_1^2 + q_2^2)^{3/2} + 3q_1^2(q_1^2 + q_2^2)^{1/2}}{(q_1^2 + q_2^2)^3} &
    \frac{3q_1q_2}{(q_1^2 + q_2^2)^{5/2}}\\
    \frac{3q_1q_2}{(q_1^2 + q_2^2)^{5/2}} &
    \frac{(q_1^2 + q_2^2)^{3/2} + 3q_2^2(q_1^2 + q_2^2)^{1/2}}{(q_1^2 + q_2^2)^3}
  \end{pmatrix}\\
  & =
  \begin{pmatrix}
    \frac{4q_1^2 + q_2^2}{(q_1^2 + q_2^2)^{5/2}} &
    \frac{3q_1q_2}{(q_1^2 + q_2^2)^{5/2}}\\
    \frac{3q_1q_2}{(q_1^2 + q_2^2)^{5/2}} &
    \frac{q_1^2 + 4q_2^2}{(q_1^2 + q_2^2)^{5/2}}
  \end{pmatrix}\\
  & =
  \frac{1}{\|q\|_2^{5/2}}
  \begin{pmatrix}
    4q_1^2 + q_2^2 &
    3q_1q_2\\
    3q_1q_2 &
    q_1^2 + 4q_2^2
  \end{pmatrix}
\end{align*}
